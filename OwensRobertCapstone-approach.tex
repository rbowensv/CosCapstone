\documentclass[11pt]{article}
\usepackage[
backend=biber,
style=ieee,
sorting=none
]{biblatex}
\usepackage[utf8]{inputenc}
\usepackage{setspace}
\usepackage{titling}
\addbibresource{OwensRobertCapstone.bib}

\title{iBeacon Proximity Beacons: A Solution to Wandering}
\author{Robert B. Owens V}
\date{\small Capstone Project\\[.5em]
\small Fall 2017}

\begin{document}

\maketitle

\section{Approach}
Recall that this work is optimizing the balance between privacy, reliability and safety. To help maintain a feeling of privacy the iBeacon proximity beacons are used only to broadcast an approximate location, and this data is only used to determine if the resident is in the area. This decision was made because residents interviewed discussed a desire to maintain privacy and not be tracked\cite{robinson}. This project is also focused on maintaining privacy by accessing as little information as needed for it to function.

The broadcasting beacon is embedded in the residents cane for this prototype because it is an object that they are required to have and can not accidentally leave it behind; for implementation an object would need to be chosen for each individual. The chosen object should be something the resident is least likely to leave without. This layer of customization makes our solution a more encompassing approach. For this project I used BlueCats proximity beacons\cite{bluecats} because of they were available; any iBeacon proximity beacons can be used in a deployment provided they are capable of transmitting far enough for your space. For a residential home environment most beacons will be suitable and this is the target space we developed for. For a receiver we used a Raspberry Pi 3 because it has built in bluetooth, wi-fi, and a friendly space to develop in. The raspberry Pi 3 is also affordable and readily available. When it came time to deploy it also had a small power requirement and a small footprint to store it in a space. 

To help preserve autonomy and prevent false wandering reports it includes a delay in its customizable settings. This allows the caregiver to specify an amount of time that the resident should be missing for before an alert is sent. To notify the caregiver when the resident has left, we use pythons e-mail library. This e-mail library can send an e-mail to the specified address of the caregiver, which these days most people can access on the go in the case of immediate response being necessary. The application knows when a resident has left by scanning for iBeacon packets at regular intervals, determined by the user.  When scanning, the receiver generates a list of packets that it heard during the scan time. For each packet the application compares the known identifier for the residents beacon;  if the residents beacon is not found in the collection, the application knows the resident is absent and starts a timer. The length of the timer is determined by the user, and is used to prevent false alerts. If the resident returns before the timer expires then no alert is sent; otherwise, the application creates an e-mail alert and sends it to the care provider. This customization allows for a solution that works best for each user. 

In addition to the delay our solution features Google Calendar integration to allow the resident to specify times when they are expected to be leaving the space. Google Calendar is easy to use and is the most widely used calendar service\cite{henry} so it made sense to make use of it. By using Google Calendar instead of a local management of events, residents or care providers can manage events from anywhere they can access the internet. Our solution makes use of the free/busy query function of Google Calendar. The free/busy query provides the calendar with a time and asks if the user has an event during that time. If they do have an event the query returns the times that the user is busy, which our solution uses to prevent notifying the care provider of an absence that is expected. By using the free/busy query no information about the event in question is returned beyond the time it takes place, preserving the privacy of the users information.After the information is no longer needed it is deleted. The returned time is further used to determine when the device should begin checking for the resident again. An example being a trip to see family, if the resident is not expected to return for a few days, the device does not need to scan for them until the appointed time of expected return. Until the resident is expected to return the device will conserve power and no longer scan for beacons.  The care provider can also make use of the calendar to note when they expect the resident to be absent.



 
 


\printbibliography

\end{document}