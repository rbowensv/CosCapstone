\documentclass[11pt]{article}
\usepackage[
backend=biber,
style=ieee,
sorting=ynt
]{biblatex}
\usepackage[utf8]{inputenc}
\usepackage[a4paper, margin=1in]{geometry}
\usepackage{setspace}
\usepackage{titling}
\addbibresource{OwensRobertCapstone.bib}

\title{JavaScript Programming Language}
\author{Robert B. Owens V}
\date{\small Capstone Project\\[.5em]
\small Fall 2017}

\begin{document}

\maketitle

\section{Related Work}
	This work on dementia related wandering is motivated by a call to action from the Maine Policy Review \cite{MPR}. The call to action discusses the rising elderly population and the demographic shift Maine is facing as well as what needs to be done to help this growing community. It further points out that the University of Maine’s mission statement is to ``advance learning and discovery … while addressing complex challenges and opportunities of the 21st century’’ making them ``well poised to respond to the aging demographic’’\cite{MPR}. The iBeacon approach is inspired by research I began at the Virtual Environment and Multimodal Interaction Lab (VEMI) at the University of Maine involving the use of iBeacons to help people navigate an indoor space without vision. Other authors have studied iBeacons \cite{chawathe,ng,schmalenstroeer} as well as solutions to dementia related wandering \cite{bail,robinson,sposaro} but none have looked at using iBeacons as a solution. 

 


\printbibliography

\end{document}